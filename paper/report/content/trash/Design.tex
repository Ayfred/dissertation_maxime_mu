\chapter{Design}
\label{chap:Design}

At the beginning of each chapter, a description should introduce the reader to the content of the chapter. The description should explain to the reader the layout of the chapter, the contribution that the chapter makes to the overall dissertation and the contribution of the individual sections towards the overall chapter.


\section{Problem Formulation}
\label{sec:ProblemFormulation}

This section should provide the reader with an overall description of the problem that will be addressed in the dissertation. In contrast to a generic discussion of the dissertation topic in the Introduction chapter, this section should provide a detailed discussion of the problem that has been identified based on the existing work that has been discussed in the preceeding chapter. 

In some dissertations, it may make sense to convert this section into a short chapter of its own which follows the discussion of the existing work and preceeds the discussion of the work of the dissertation.

\subsection{Identified Challenges}
This section should present a short description of the gaps in the existing work and the relationship of these gaps to the work described in this dissertation.

\subsection{Proposed Work}
This section should provide a thorough description of the problem and an overview of the work proposed to address the problem.


\section{Overview of the Design}
\label{sec:OverviewOfDesign}
A description of the approach that addresses the problem identified above.


\section{Summary}
\label{sec:SummaryDesign}

Every chapter aside from the first and last chapter should conclude with a summary. 