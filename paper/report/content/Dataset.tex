\chapter{Datasets}
\label{chap:Datasets}

In this project, two datasets will be considered to ensure to validation of the results.
Given the ethical question around the manipulation of hospitals' data, the datasets used in this paper are publicly available and accessible via Kaggle.

\section{First Dataset}

The first dataset consists of 349 patients' health records in a CSV format (\cite{uom190346a_disease_symptoms_2021}). It contains 10 features including the patient's age, disease, and specific symptoms that are associated with the disease. 


This dataset quite is small given the number of patients. It will be used to generate the synthetic data, which is compared with the original data to perform statistical comparisons and also to train machine learning models to predict disease based on the other features.

It will be interesting to see how LLMs and GAN models can perform on such datasets.



\section{Second Dataset}

The second dataset consists of more than 4412 patients' health records (\cite{palivela_patient_treatment_2021}). It includes 11 features containing various haematological parameters measured in individuals, such as hematocrit, haemoglobin, erythrocyte count, leucocyte count, thrombocyte count, mean corpuscular haemoglobin (MCH), mean corpuscular haemoglobin concentration (MCHC), mean corpuscular volume (MCV), age, sex, and source of the sample (inpatient or outpatient). 

The second dataset is considered because the first dataset used is relatively small and might produce inaccurate generated results. Having a second dataset helps justify the results previously generated and it will be interesting to see how leveraging LLM power to provide relevant information based on such a large amount of information given initially.