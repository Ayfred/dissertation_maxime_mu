\chapter{Research Questions}



\noindent \texttt{Q1: How can LLMs be used to generate synthetic tabular data?}

Due to their text-generative nature, LLMs could potentially create a synthetic dataset based on the context we are giving to them. It will be interesting to see how LLMs can perform such tasks and the data being generated differs from the initial data.
Prompt engineering will be employed to see how the context is provided to the LLMs, and how the latter can guide the LLMs in generating relevant data.


%\noindent \texttt{Q2: What types of synthetic data can be generated using LLMs?}

%In other words, what is the output of a LLM? Can they directly provide data in a tabular format or is post-processing required? LLMs could possibly generate synthetic data with varying levels of complexity, for example, it might introduce new diseases that were not present initially in the data. They could probably capture the hierarchical relationships and multi-dimensional data of the initial data.

\vspace{1cm}

\noindent \texttt{Q2: How can the quality and realism of synthetic data be evaluated?}

Statistical tests will be considered to compare the distribution of variables in the synthetic data with the real data. Several metrics will be used such as mean, standard deviation, correlation between features, t-SNE graph plots, Principal Component analysis, and frequency. 
More domain-specific evaluation will also be considered, in our case the healthcare domain, we can assess the plausibility of generated diagnoses or diseases.

%\noindent \texttt{Q4: What are the ethical considerations and potential biases associated with using LLMs for data generation?}

%A comparison will be made between the generated data and the initial data to look for any duplicate data. Duplicate generated data with the initial data means that the LLM was not able to synthesize new data but copied the original data which means that patient sensitive information from real data could be leaked. It is thus crucial to explore privacy-preserving techniques to look for duplicated data.


%\noindent \section{Research Objectives}

% Develop a framework for using LLMs to generate synthetic tabular data

% Design and implement an LLM-based data generation pipeline.

% Evaluate the quality and realism of synthetic data generated by the LLM.

% Analyse the ethical and societal implications of using LLMs for data generation.